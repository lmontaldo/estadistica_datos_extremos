% Options for packages loaded elsewhere
\PassOptionsToPackage{unicode}{hyperref}
\PassOptionsToPackage{hyphens}{url}
%
\documentclass[
  12pt]{article}
\usepackage{amsmath,amssymb}
\usepackage{iftex}
\ifPDFTeX
  \usepackage[T1]{fontenc}
  \usepackage[utf8]{inputenc}
  \usepackage{textcomp} % provide euro and other symbols
\else % if luatex or xetex
  \usepackage{unicode-math} % this also loads fontspec
  \defaultfontfeatures{Scale=MatchLowercase}
  \defaultfontfeatures[\rmfamily]{Ligatures=TeX,Scale=1}
\fi
\usepackage{lmodern}
\ifPDFTeX\else
  % xetex/luatex font selection
\fi
% Use upquote if available, for straight quotes in verbatim environments
\IfFileExists{upquote.sty}{\usepackage{upquote}}{}
\IfFileExists{microtype.sty}{% use microtype if available
  \usepackage[]{microtype}
  \UseMicrotypeSet[protrusion]{basicmath} % disable protrusion for tt fonts
}{}
\makeatletter
\@ifundefined{KOMAClassName}{% if non-KOMA class
  \IfFileExists{parskip.sty}{%
    \usepackage{parskip}
  }{% else
    \setlength{\parindent}{0pt}
    \setlength{\parskip}{6pt plus 2pt minus 1pt}}
}{% if KOMA class
  \KOMAoptions{parskip=half}}
\makeatother
\usepackage{xcolor}
\usepackage{color}
\usepackage{fancyvrb}
\newcommand{\VerbBar}{|}
\newcommand{\VERB}{\Verb[commandchars=\\\{\}]}
\DefineVerbatimEnvironment{Highlighting}{Verbatim}{commandchars=\\\{\}}
% Add ',fontsize=\small' for more characters per line
\usepackage{framed}
\definecolor{shadecolor}{RGB}{248,248,248}
\newenvironment{Shaded}{\begin{snugshade}}{\end{snugshade}}
\newcommand{\AlertTok}[1]{\textcolor[rgb]{0.94,0.16,0.16}{#1}}
\newcommand{\AnnotationTok}[1]{\textcolor[rgb]{0.56,0.35,0.01}{\textbf{\textit{#1}}}}
\newcommand{\AttributeTok}[1]{\textcolor[rgb]{0.13,0.29,0.53}{#1}}
\newcommand{\BaseNTok}[1]{\textcolor[rgb]{0.00,0.00,0.81}{#1}}
\newcommand{\BuiltInTok}[1]{#1}
\newcommand{\CharTok}[1]{\textcolor[rgb]{0.31,0.60,0.02}{#1}}
\newcommand{\CommentTok}[1]{\textcolor[rgb]{0.56,0.35,0.01}{\textit{#1}}}
\newcommand{\CommentVarTok}[1]{\textcolor[rgb]{0.56,0.35,0.01}{\textbf{\textit{#1}}}}
\newcommand{\ConstantTok}[1]{\textcolor[rgb]{0.56,0.35,0.01}{#1}}
\newcommand{\ControlFlowTok}[1]{\textcolor[rgb]{0.13,0.29,0.53}{\textbf{#1}}}
\newcommand{\DataTypeTok}[1]{\textcolor[rgb]{0.13,0.29,0.53}{#1}}
\newcommand{\DecValTok}[1]{\textcolor[rgb]{0.00,0.00,0.81}{#1}}
\newcommand{\DocumentationTok}[1]{\textcolor[rgb]{0.56,0.35,0.01}{\textbf{\textit{#1}}}}
\newcommand{\ErrorTok}[1]{\textcolor[rgb]{0.64,0.00,0.00}{\textbf{#1}}}
\newcommand{\ExtensionTok}[1]{#1}
\newcommand{\FloatTok}[1]{\textcolor[rgb]{0.00,0.00,0.81}{#1}}
\newcommand{\FunctionTok}[1]{\textcolor[rgb]{0.13,0.29,0.53}{\textbf{#1}}}
\newcommand{\ImportTok}[1]{#1}
\newcommand{\InformationTok}[1]{\textcolor[rgb]{0.56,0.35,0.01}{\textbf{\textit{#1}}}}
\newcommand{\KeywordTok}[1]{\textcolor[rgb]{0.13,0.29,0.53}{\textbf{#1}}}
\newcommand{\NormalTok}[1]{#1}
\newcommand{\OperatorTok}[1]{\textcolor[rgb]{0.81,0.36,0.00}{\textbf{#1}}}
\newcommand{\OtherTok}[1]{\textcolor[rgb]{0.56,0.35,0.01}{#1}}
\newcommand{\PreprocessorTok}[1]{\textcolor[rgb]{0.56,0.35,0.01}{\textit{#1}}}
\newcommand{\RegionMarkerTok}[1]{#1}
\newcommand{\SpecialCharTok}[1]{\textcolor[rgb]{0.81,0.36,0.00}{\textbf{#1}}}
\newcommand{\SpecialStringTok}[1]{\textcolor[rgb]{0.31,0.60,0.02}{#1}}
\newcommand{\StringTok}[1]{\textcolor[rgb]{0.31,0.60,0.02}{#1}}
\newcommand{\VariableTok}[1]{\textcolor[rgb]{0.00,0.00,0.00}{#1}}
\newcommand{\VerbatimStringTok}[1]{\textcolor[rgb]{0.31,0.60,0.02}{#1}}
\newcommand{\WarningTok}[1]{\textcolor[rgb]{0.56,0.35,0.01}{\textbf{\textit{#1}}}}
\usepackage{longtable,booktabs,array}
\usepackage{calc} % for calculating minipage widths
% Correct order of tables after \paragraph or \subparagraph
\usepackage{etoolbox}
\makeatletter
\patchcmd\longtable{\par}{\if@noskipsec\mbox{}\fi\par}{}{}
\makeatother
% Allow footnotes in longtable head/foot
\IfFileExists{footnotehyper.sty}{\usepackage{footnotehyper}}{\usepackage{footnote}}
\makesavenoteenv{longtable}
\usepackage{graphicx}
\makeatletter
\def\maxwidth{\ifdim\Gin@nat@width>\linewidth\linewidth\else\Gin@nat@width\fi}
\def\maxheight{\ifdim\Gin@nat@height>\textheight\textheight\else\Gin@nat@height\fi}
\makeatother
% Scale images if necessary, so that they will not overflow the page
% margins by default, and it is still possible to overwrite the defaults
% using explicit options in \includegraphics[width, height, ...]{}
\setkeys{Gin}{width=\maxwidth,height=\maxheight,keepaspectratio}
% Set default figure placement to htbp
\makeatletter
\def\fps@figure{htbp}
\makeatother
\setlength{\emergencystretch}{3em} % prevent overfull lines
\providecommand{\tightlist}{%
  \setlength{\itemsep}{0pt}\setlength{\parskip}{0pt}}
\setcounter{secnumdepth}{-\maxdimen} % remove section numbering
\newlength{\cslhangindent}
\setlength{\cslhangindent}{1.5em}
\newlength{\csllabelwidth}
\setlength{\csllabelwidth}{3em}
\newlength{\cslentryspacingunit} % times entry-spacing
\setlength{\cslentryspacingunit}{\parskip}
\newenvironment{CSLReferences}[2] % #1 hanging-ident, #2 entry spacing
 {% don't indent paragraphs
  \setlength{\parindent}{0pt}
  % turn on hanging indent if param 1 is 1
  \ifodd #1
  \let\oldpar\par
  \def\par{\hangindent=\cslhangindent\oldpar}
  \fi
  % set entry spacing
  \setlength{\parskip}{#2\cslentryspacingunit}
 }%
 {}
\usepackage{calc}
\newcommand{\CSLBlock}[1]{#1\hfill\break}
\newcommand{\CSLLeftMargin}[1]{\parbox[t]{\csllabelwidth}{#1}}
\newcommand{\CSLRightInline}[1]{\parbox[t]{\linewidth - \csllabelwidth}{#1}\break}
\newcommand{\CSLIndent}[1]{\hspace{\cslhangindent}#1}
\ifLuaTeX
\usepackage[bidi=basic]{babel}
\else
\usepackage[bidi=default]{babel}
\fi
\babelprovide[main,import]{spanish}
% get rid of language-specific shorthands (see #6817):
\let\LanguageShortHands\languageshorthands
\def\languageshorthands#1{}
\usepackage{makeidx}
\makeindex
\usepackage{graphicx}
\usepackage{tikz}
\usepackage{atbegshi}

\AtBeginDocument{
    \AtBeginShipoutNext{
        \AtBeginShipoutUpperLeft{
            \put(\dimexpr\paperwidth/2-\textwidth/2\relax, -650){
                \makebox[\textwidth]{
                    \includegraphics[width=0.45\textwidth]{cure_udelar.png}  % Adjust width as needed
                    \hfill
                    \includegraphics[width=0.405\textwidth]{logoMEDIA.jpeg} % Make it 90% smaller
                }
            }
        }
    }
}
\ifLuaTeX
  \usepackage{selnolig}  % disable illegal ligatures
\fi
\IfFileExists{bookmark.sty}{\usepackage{bookmark}}{\usepackage{hyperref}}
\IfFileExists{xurl.sty}{\usepackage{xurl}}{} % add URL line breaks if available
\urlstyle{same}
\hypersetup{
  pdftitle={Entrega: curso de datos extremales},
  pdfauthor={Laura Montaldo, CI: 3.512.962-7},
  pdflang={es},
  hidelinks,
  pdfcreator={LaTeX via pandoc}}

\title{Entrega: curso de datos extremales}
\author{Laura Montaldo, CI: 3.512.962-7}
\date{2024-02-24}

\begin{document}
\maketitle

\newpage

\thispagestyle{empty}

\maketitle

\newpage

\tableofcontents

\newpage

\hypertarget{resumen}{%
\section{Resumen}\label{resumen}}

Your abstract goes here.

\newpage

\hypertarget{motivaciuxf3n-y-objetivo-del-estudio}{%
\section{Motivación y objetivo del
estudio}\label{motivaciuxf3n-y-objetivo-del-estudio}}

Los índices de \(S\&P\) son una familia de índices de renta
variable\footnote{En inglés se llaman equity indices} diseñados para
medir el rendimiento del mercado de acciones en Estados Unidos que
cotizan en bolsas estadounidenses. Ésta familia de índices está
compuesta por una amplia variedad de índices basados en tamaño, sector y
estilo. Los índices están ponderados por el criterio
\textit{float-adjusted market capitalization} (FMC). Además, se disponen
de índices ponderados de manera equitativa y con límite de
capitalización de mercado, como es el caso del \(S\&P\:500\). Este este
sentido, el \(S\&P 500\) entraría en el conjunto de índices ponderados
por capitalización bursátil ajustada a la flotación (ver
\href{http://www.overleaf.com}{\textcolor{blue}{$S\&P$ Dow Jones Indices}}).
El mismo mide el rendimiento del segmento de gran capitalización del
mercado estadounidense. Es considerado como un indicador representativo
del mercado de renta variable de los Estados Unidos, y está compuesto
por 500 empresas constituyentes.

Se busca crear un indicador de una posible crisis bursátil. Como
variable de referencia de toma la relación de precios al cierre de ayer
sobre la de hoy

\begin{equation}
Indicador_t=\frac{Precio_{t-1}}{Precio_t},\quad\text{para}\; t=1,...,T \label{eq:ind}
\end{equation} \vspace{0.5cm}

Interpretación del Indicador:

\begin{itemize}
\item Si el $Indicador_t$    $\leq$ 1, el precio de cierre de hoy es mayor o igual que el de ayer, lo cual podría ser considerado una señal positiva.
\item Si el $Indicador_t$ > 1, el precio de cierre de hoy es menor que el de ayer, lo cual podría considerarse una señal de alerta.
\end{itemize}

\vspace{1cm}

En las siguiente figura @ref(fig:plot1) se muestra la evolución
histórica desde la fecha 03/01/1928 hasta 08/12/2023 del precio al
cierre del día del indicar S\&P 500.

\includegraphics{extremales_files/figure-latex/plot1-1.pdf}

\includegraphics{extremales_files/figure-latex/plot2-1.pdf} \newpage

\hypertarget{marco-teuxf3rico}{%
\section{Marco Teórico}\label{marco-teuxf3rico}}

\hypertarget{teoruxeda-asintuxf3tica-cluxe1sica-y-las-distribuciones-extremales-y-sus-dominios-de-atracciuxf3n}{%
\subsection{Teoría asintótica clásica y las distribuciones extremales y
sus dominios de
atracción}\label{teoruxeda-asintuxf3tica-cluxe1sica-y-las-distribuciones-extremales-y-sus-dominios-de-atracciuxf3n}}

Siguiendo a Perera, Segura, y Crisci (2021) se dice que tenemos datos
extremos cuando cada dato corresponde al máximo o mínimo de varios
registros. Son un caso particular de evento raro o gran desviación
respecto a la media.

Asumiremos que nuestros datos son \(iid\) (independientes e
idénticamente distribuidos, son dos suposiciones juntas). Esta doble
suposición suele no ser realista en aplicaciones concretas (ninguna de
sus dos componentes, incluso) pero para comenzar a entender la teoría
clásica, la utilizaremos por un tiempo.

Si tenemos datos \(X_1,...,X_n\) \(iid\) con distribución \(F\),
entonces \(X_n^* = max (X_1,...,X_n)\) tiene distribución \(F_n^*\) dada
por \(F_n^* (t)= F(t)_n\). Si conocemos la distribución \(F\)
conoceríamos la distribución \(F_n^*\) , pero en algunos casos la
lectura que queda registrada es la del dato máximo y no la de cada
observación que dio lugar al mismo, por lo que a veces ni siquiera es
viable estimar \(F\). Pero aún en los casos en que \(F\) es conocida o
estimable, si \(n\) es grande, la fórmula de \(F_n^*\) puede resultar
prácticamente inmanejable. En una línea de trabajo similar a la que
aporta el Teorema Central del Límite en la estadística de valores
medios, un teorema nos va a permitir aproximar \(F_n^*\) por
distribuciones más sencillas. Este es el Teorema de
Fischer-Tippet-Gnedenko (FTG, para abreviar) que presentaremos en breve.

Como \(X_1,...,X_n\) iid, definimos \(Y_i = -X_i\) para todo valor de
\(i\), entonces \(Y_1,...,Y_n\) iid y además
\(min(X_1,...,X_n) = - max(Y_1,...,Y_n)\) la teoría asintótica de los
mínimos de datos iid se reduce a la de los máximos, razón por la que nos
concentramos aquí en estudiar el comportamiento asintótico de los
máximos exclusivamente.

\newpage

\hypertarget{definiciuxf3n-1-las-distribuciones-extremales}{%
\subsubsection{Definición 1: Las distribuciones
extremales}\label{definiciuxf3n-1-las-distribuciones-extremales}}

Las distribuciones extremales son tres: la distribución de Gumbel; la
distribución de Weibull; la distribución de Fréchet.

\hypertarget{distribuciuxf3n-de-gumbel}{%
\paragraph{Distribución de Gumbel}\label{distribuciuxf3n-de-gumbel}}

Se dice que una variable tiene distribución de Gumbel si su distribución
es:

\[ \Lambda(x) = exp\{-e^{-x}\} \quad\text{para todo}\; x \;\text{real} \]

Cuando tomamos los máximos de variables no acotadas pero que tienen
colas livianas (ej. la distribución tiene probabilidades muy bajas de
tomar valores lejos de la media) los mismos convergen a una distribución
asintótica extremal de Gumbel.

Para simular distribuciones de Gumbel, utilizamos el paquete
\textbf{evd} de Stephenson (2002) y en particular la función
\textbf{pgumbel}. Partiendo de una simulación de números aleatorios,
para un secuencia de 1000 números entre \([-10,10]\), se tienen las
siguientes figuras @ref(fig:gumbel\_plots) relativas a la CDF y PDF de
la distribución Gumbel.

\begin{figure}
\centering
\includegraphics{extremales_files/figure-latex/gumbel_plots-1.pdf}
\caption{CDF and PDF for Gumbel distribution.}
\end{figure}

Si calculamos el valor esperado y el desvío estandard de estos valores
observados y tenemos una muestra lo suficientemente grande, podremos
comparar los resultados con los esperados de forma teórica.

\begin{Shaded}
\begin{Highlighting}[]
\CommentTok{\# Podemos simular 100 datos aleatorios de una distribución Gumbel}
\NormalTok{GumbelAleatorio}\OtherTok{\textless{}{-}}\FunctionTok{rgumbel}\NormalTok{(}\DecValTok{100}\NormalTok{)}
\FunctionTok{plot}\NormalTok{(}\FunctionTok{density}\NormalTok{(GumbelAleatorio))}
\end{Highlighting}
\end{Shaded}

\includegraphics{extremales_files/figure-latex/unnamed-chunk-12-1.pdf}

\begin{Shaded}
\begin{Highlighting}[]
\SpecialCharTok{{-}}\FunctionTok{digamma}\NormalTok{(}\DecValTok{1}\NormalTok{) }\CommentTok{\# Constante de Euler{-}Mascheroni}
\end{Highlighting}
\end{Shaded}

\begin{verbatim}
## [1] 0.5772157
\end{verbatim}

\begin{Shaded}
\begin{Highlighting}[]
\FunctionTok{mean}\NormalTok{(}\FunctionTok{rgumbel}\NormalTok{(}\DecValTok{1000}\NormalTok{))}
\end{Highlighting}
\end{Shaded}

\begin{verbatim}
## [1] 0.5563899
\end{verbatim}

\begin{Shaded}
\begin{Highlighting}[]
\FunctionTok{sd}\NormalTok{(}\FunctionTok{rgumbel}\NormalTok{(}\DecValTok{1000}\NormalTok{))}
\end{Highlighting}
\end{Shaded}

\begin{verbatim}
## [1] 1.236503
\end{verbatim}

\hypertarget{distribuciuxf3n-de-weibull}{%
\paragraph{Distribución de Weibull}\label{distribuciuxf3n-de-weibull}}

Se dice que una variable tiene distribución de Weibull de orden
\(\alpha>0\) si su distribución es:

\[\Psi_{\alpha}(x)=\begin{cases}
exp{-(-x)^{\alpha}} & si\;x<0\\
1 & \text{en otro caso}
\end{cases}\] Recordemos que cuando tomamos los máximos de las variables
\(iid\) con un rango acotado, la distribución resultante por la cual se
puede aproximar es la de Weibull. En este caso, y en el resto del LAB,
exp() y e son la función exponencial.

Por una única vez, calculemos la distribución de forma ``manual'' en el
R para convencernos de la forma de la función de distribución de Weibull
(\(\Psi\)). Para eso generaremos un vector auxiliar de valores \(x\) y
la distribución (\(F(x)\)). En R la definición de la distribución es
sutilmente diferente a la que vimos en el teórico (definida para
positivos), pero totalmente convertible con dos cambios de signo. La
función que calcula la probabilidad de una distribución Weibull es
\textbf{pweibull()}. Pueden ver la definición de R utilizando
help(pweibull) o ?pweibull.En R podemos saber la forma y valores de esta
distribución con una función implementada en un paquete base \{stats\}.
La función es pweibull y lleva como argumentos un vector de cuantiles (
q ), un argumento de forma ( shape ) y otro de escala ( scale ).
Recordemos que la función plot utiliza 2 argumentos centrales ( x e y )
y podemos fijar los límites del gráfico ( xlim e ylim), el tipo de
gráfico ( type) y las etiquetas de los ejes X e Y ( xlab e ylab).

Primero generaremos un vector de numeros auxiliares equiespaciados y lo
nombraremos (``x\_aux''). Luego definiremos un orden (alpha=α ) de la
Weibull y graficaremos la función.

\includegraphics{extremales_files/figure-latex/unnamed-chunk-16-1.pdf}
Veamos ahora la forma de un par de distribuciones cambiando el parámetro
de orden (α ), que en la función pweibull de R se nombra como shape y
que define el orden de la distribución.

\includegraphics{extremales_files/figure-latex/unnamed-chunk-17-1.pdf}
En R podemos también generar numeros aleatórios (técnicamente
pseudo-aleatorios) de una distribución extremal. Estos simuladores de
números aleatórios son útiles para comparar contra distribuciones nulas,
generar modelos sintéticos para probar algorítmos, etc\ldots{} Para lxs
que venimos de la rama mas aplicada, muchas veces nos ayudan a entender
como funcionan los modelos y a verificar si nuestra intuición es
acertada respecto a la escala de ajuste de los parámetros entre otras
útiles. Generaremos 2 series de 1000 números aleatórios con la función
rweibull, que tiene como parámetro el número de datos que se necesitan y
la forma (shape) de la distribución. Luego haremos un grafico con la
densidad empírica (esto es similar a un histograma) de estos vectores.

\includegraphics{extremales_files/figure-latex/unnamed-chunk-18-1.pdf}

\hypertarget{distribuciuxf3n-de-fruxe9chet}{%
\paragraph{Distribución de
Fréchet}\label{distribuciuxf3n-de-fruxe9chet}}

Se dice que una variable tiene distribución de Fréchet de orden
\(\alpha>0\) si su distribución es:

\[
\Phi_{\alpha}(x)=\begin{cases}
exp\{-x^{-\alpha}\} & si\;x>0\\
0 & \text{en otro caso}
\end{cases}
\]

Esta tercera clase de variables incluyen a las distribuciones no
acotadas, pero de colas pesadas. Es decir que tienen una probabilidad
alta de presentar valores alejados de la media o la mediana (ej. la
Cauchy). En estos casos, la distribución de sus máximos es la Frechet.
Grafiquemos esta distribución para dos valores diferentes de \(\alpha\).

\begin{Shaded}
\begin{Highlighting}[]
\NormalTok{x\_aux}\OtherTok{\textless{}{-}} \FunctionTok{seq}\NormalTok{(}\SpecialCharTok{{-}}\DecValTok{10}\NormalTok{,}\DecValTok{10}\NormalTok{, }\AttributeTok{length=}\DecValTok{1000}\NormalTok{)}

\FunctionTok{par}\NormalTok{(}\AttributeTok{mfrow=}\FunctionTok{c}\NormalTok{(}\DecValTok{3}\NormalTok{,}\DecValTok{1}\NormalTok{), }\AttributeTok{mar=}\FunctionTok{c}\NormalTok{(}\DecValTok{5}\NormalTok{,}\DecValTok{4}\NormalTok{,}\DecValTok{3}\NormalTok{,}\DecValTok{1}\NormalTok{))}
\FunctionTok{plot}\NormalTok{(}\FunctionTok{seq}\NormalTok{(}\SpecialCharTok{{-}}\DecValTok{10}\NormalTok{,}\DecValTok{10}\NormalTok{,}\AttributeTok{length=}\DecValTok{100}\NormalTok{), }\FunctionTok{pfrechet}\NormalTok{(}\AttributeTok{q=}\FunctionTok{seq}\NormalTok{(}\SpecialCharTok{{-}}\DecValTok{10}\NormalTok{,}\DecValTok{10}\NormalTok{,}\AttributeTok{length=}\DecValTok{100}\NormalTok{), }\AttributeTok{shape=}\DecValTok{2}\NormalTok{, }\AttributeTok{scale=}\DecValTok{1}\NormalTok{) ,}\AttributeTok{xlim=}\FunctionTok{c}\NormalTok{(}\SpecialCharTok{{-}}\DecValTok{2}\NormalTok{,}\DecValTok{10}\NormalTok{), }\AttributeTok{type=}\StringTok{"l"}\NormalTok{, }\AttributeTok{ylab=}\StringTok{"F(x)"}\NormalTok{, }\AttributeTok{xlab=}\StringTok{"x"}\NormalTok{, }\AttributeTok{main=}\StringTok{"Frechet"}\NormalTok{)}
\FunctionTok{lines}\NormalTok{(}\FunctionTok{seq}\NormalTok{(}\SpecialCharTok{{-}}\DecValTok{10}\NormalTok{,}\DecValTok{10}\NormalTok{,}\AttributeTok{length=}\DecValTok{100}\NormalTok{), }\FunctionTok{pfrechet}\NormalTok{(}\AttributeTok{q=}\FunctionTok{seq}\NormalTok{(}\SpecialCharTok{{-}}\DecValTok{10}\NormalTok{,}\DecValTok{10}\NormalTok{,}\AttributeTok{length=}\DecValTok{100}\NormalTok{), }\AttributeTok{shape=}\FloatTok{1.1}\NormalTok{, }\AttributeTok{scale=}\DecValTok{1}\NormalTok{),}\AttributeTok{col=} \StringTok{"red"}\NormalTok{)}

\FunctionTok{plot}\NormalTok{(x\_aux, }\FunctionTok{dfrechet}\NormalTok{(}\AttributeTok{x=}\NormalTok{x\_aux, }\AttributeTok{shape=}\DecValTok{2}\NormalTok{, }\AttributeTok{scale=}\DecValTok{1}\NormalTok{, }\AttributeTok{log =} \ConstantTok{FALSE}\NormalTok{) ,}\AttributeTok{xlim=}\FunctionTok{c}\NormalTok{(}\SpecialCharTok{{-}}\DecValTok{2}\NormalTok{,}\DecValTok{10}\NormalTok{), }\AttributeTok{type=}\StringTok{"l"}\NormalTok{, }\AttributeTok{ylab=}\StringTok{"f(x)"}\NormalTok{, }\AttributeTok{xlab=}\StringTok{"x"}\NormalTok{)}
\FunctionTok{lines}\NormalTok{(x\_aux, }\FunctionTok{dfrechet}\NormalTok{(}\AttributeTok{x=}\NormalTok{x\_aux, }\AttributeTok{shape=}\FloatTok{1.1}\NormalTok{, }\AttributeTok{scale=}\DecValTok{1}\NormalTok{, }\AttributeTok{log =} \ConstantTok{FALSE}\NormalTok{), }\AttributeTok{col=}\StringTok{"red"}\NormalTok{)}
\end{Highlighting}
\end{Shaded}

\includegraphics{extremales_files/figure-latex/unnamed-chunk-19-1.pdf}

.

\newpage

\hypertarget{teorema-1-relaciones-entre-las-versiones-standard-de-las-distribuciones-extremales}{%
\subparagraph{Teorema 1: Relaciones entre las versiones standard de las
distribuciones
extremales}\label{teorema-1-relaciones-entre-las-versiones-standard-de-las-distribuciones-extremales}}

\(X\) tiene distribución \(\Phi_{\alpha}(x)\) si y sólo si \((-1/X)\)
tiene distribución \(\Psi_{\alpha}(x)\) si y sólo si \(log(X^{\alpha})\)
tiene distribución \(\Lambda\).

\hypertarget{teorema-2-algunos-datos-de-las-distribuciones-extremales}{%
\subparagraph{Teorema 2: Algunos datos de las distribuciones
extremales}\label{teorema-2-algunos-datos-de-las-distribuciones-extremales}}

\hypertarget{parte-1}{%
\subparagraph{Parte 1}\label{parte-1}}

Si \(X\) tiene distribución \(\Lambda^{(\mu,\beta)}\) entonces tiene:

\begin{itemize}
  \item[a)] Valor esperado: $E(X) = \mu + \beta\gamma$, donde $\gamma$ es la constante de Euler-Mascheroni, cuyo valor aproximado es $0.5772156649$.
  \item[b)] Moda: $\mu$
  \item[c)] Mediana: $\mu - \beta \log(\log 2) \approx \mu - 0.36651 \beta$.
  \item[d)] Desviación estándar: $\beta \pi \sqrt{6} \approx 1.2825 \beta$.
  \item[e)] Si $X^+ = \max(X,0)$, entonces $E(X+k)$ es finito para todo valor de $k$ natural.
  \item[f)] Para simular computacionalmente $X$, se puede tomar $U$ uniforme en $(0,1)$ y hacer $X = \mu - \beta \log(-\log U)$.
\end{itemize}

\hypertarget{parte-2}{%
\paragraph{Parte 2}\label{parte-2}}

Si \(X\) tiene distribución \(\Psi_{\alpha}^{(\mu,\beta)}\) entonces
tiene:

\begin{itemize}
  \item[a)] Valor esperado: $E(X) = \mu + \beta\Gamma(1+1/\alpha)$.
  \item[b)] Moda: $\mu$ si $\alpha\leq 1$ y $\mu-\beta\{(\alpha-1)/\alpha\}^{(1/\alpha)}$ si $\alpha>1$.
  \item[c)] Mediana: $\mu - \beta \log(2)^{(1/\alpha)}$.
  \item[d)] Desviación estándar: $\beta\{\Gamma(1+2/\alpha)-\Gamma(1+1/\alpha)^2\}^{1/2}$.
\end{itemize}

\hypertarget{parte-2-1}{%
\paragraph{Parte 2}\label{parte-2-1}}

Si \(X\) tiene una distribución \(\Phi_{\alpha}^{(\mu, \beta)}\)
entonces se tiene:

\begin{itemize}
  \item[a)] Valor esperado: $E(X) = \mu + \beta\Gamma(1-1/\alpha)$ si $\alpha > 1$, $\infty$ en caso contrario.
  \item[b)] Moda: $\mu + \beta\Gamma(1-1/\alpha)$ si $\alpha>1$.
  \item[c)] Mediana: $\mu + \beta \log(2)^{(-1/\alpha)}$.
  \item[d)] Desviación estándar: $\beta|\Gamma(1-2/\alpha)-\Gamma(1-1/\alpha)^2|$ si $\alpha>2$, $\infty$ si $1<\alpha \leq 2$.
\end{itemize}

\newpage

\hypertarget{teorema-3-fischer-tippet-gnedenko-ftg}{%
\subparagraph{Teorema 3: Fischer-Tippet-Gnedenko
(FTG)}\label{teorema-3-fischer-tippet-gnedenko-ftg}}

Si \(X_1,...,X_n\quad iid\) con distribución \(F\) ``continua'',
llamamos \(F_n^*\) a la distribución de \(max(X_1,...,X_n)\) y \(n\) es
grande, entonces existen \(\mu\) real y \(\beta>0\) tales que alguna de
las siguientes tres afirmaciones es correcta:

\begin{itemize}
  \item[1)] $F_n^*$ se puede apromixar por la distribución de $\mu+\beta Y$ con $Y$ variable con distribución $\Lambda$.
  \item[2)] Existe $\alpha>0$ tal que $F_n^*$ se puede aproximar por la distribución de $\mu+\beta Y$ con $Y$ variable con distribución $\Phi_{\alpha}$. 
  \item[3)] Existe $\alpha>0$ tal que $F_n^*$ se puede aproximar por la distribución de $\mu+\beta Y$ con $Y$ variable con distribución $\Phi_{\alpha}$.
\end{itemize}

Lo anterior equivale a decir que la distribución del máximo de datos
\textit{continuos} e \(iid\), si \(n\) es grande, puede aproximarse por
una Gumbel, una Fréchet o una Weibull. Una aproximación será válida
dependiendo de la distribución de \(F\). En este sentido, cuando \(F\)
sea normal entonces \(F_n^*\) se puede aproximar como una Gumbel. Cuando
\(F\) sea uniforme, se puede aproximar \(F_n^*\) como una Weibull y
cuando \(F\) sea Cauchy entonces \(F_n^*\) se puede aproximar por una
Fréchet.

Más precisamente, cuál de las tres aproximaciones es la aplicable
depende de la cola de \(F\) (los valores de \(F(t)\) para valores
grandes de \(t\)). En concreto, Weibull aparece cuando \(F\) es la
distribución de una variable acotada por arriba (como la Uniforme),
Gumbel para distribuciones de variables no acotadas por arriba pero con
colas muy livianas (caso Exponencial y Normal) y Fréchet para colas
pesadas (caso
Cauchy)\footnote{Si bien  la hipótesis de continuidad de $F$ no es esencial, si $F$ tiene
la distribución Binomial o Poisson, por ejemplo, no se puede aplicar ninguna de las tres aproximaciones anteriores.}.

Como consecuencia del \(FTG\) cuando se tengan datos máximos, las
distribuciones maximales podrían ser candidatas de uno de los ajustes si

\begin{itemize}
\item la cantidad de registros es lo suficientemente grande
\item los registros son $iid$ aunque con versiones más generales del $FTG$ este supuesto puede no cumplirse
\end{itemize}

Como la mayoría de tests de ajustes suponen datos \(iid\), se van a
realizar dos tests de
aleatoriedad\footnote{En inglés se expresa como \textit{randomness}} a
los datos:

\begin{itemize}
\item  Runs up and down 
\item  Spearman correlation of ranks 
\end{itemize}

Se emplea la prueba de ajuste \(\chi^2\) que requiere seleccionar una
partición más o menos arbitraria de la recta real de intervalos siendo
importante que en cada intervalo haya una cantidad lo suficientemente
importante de datos de la muestra. En este sentido, se pueden tomar como
extremos de los intervalos los quintiles empíricos de la muestra. Cabe
mencionar que este test requiere estimar parámetros por el método de
Máxia Verosimilitud Categórica.

Cabe mencionar que para este estudio la distribución de la variable a
incorporar en este estudio no tiene que ser degenerada, es decir
\(H(t)=0\) ó \(H(t)=1\).

\newpage

\hypertarget{definiciuxf3n-2-distribuciuxf3n-extremal-asintuxf3tica}{%
\subsubsection{Definición 2: Distribución extremal
asintótica}\label{definiciuxf3n-2-distribuciuxf3n-extremal-asintuxf3tica}}

Si \(X_1,...,X_n\) es \(iid\) con distribución \(F\) diremos que \(H\)
no-degenerada es la Distribución Extremal Asintótica (DEA) de
\(F\)\footnote{Lo que equivale a decir que $F$ tiene $DEA\;H$.}, si
existen dos sucesiones de números reales, \(d_n\) y \(c_n>0\), tales que
la distribución de

\begin{equation}
\frac{max(X_1,...,X_n)- d_n}{c_n}\label{eq:max}
\end{equation}

tiende a \(H\) cuando \(n\) tiende a infinito.

\hypertarget{definiciuxf3n-3-supremo-esencial-de-una-variable-aleatoria-o-distribuciuxf3n}{%
\subsubsection{Definición 3: Supremo esencial de una variable aleatoria
o
distribución}\label{definiciuxf3n-3-supremo-esencial-de-una-variable-aleatoria-o-distribuciuxf3n}}

Si \(X\) tiene distribución \(F\), se llama supremo esencial de \(X\),
denotado como \(M_X\) o, indistintamente, supremo esencial de \(F\),
denotado \(MF\) a

\begin{equation}
M_X=M_F= sup\{t / F(t)<1\}\label{eq:Mx}
\end{equation}

Observación:

\begin{itemize}
\item Si $F$ es $U(a,b)$, $M_F=b$
\item Si $F$ es $Bin(m,p)$, $M_F=m$
\item Si $F$ es Normal, Exponencial, Cauchy o Poisson, $M_F$ es infinito.
\end{itemize}

\hypertarget{teorema-4}{%
\subparagraph{Teorema 4}\label{teorema-4}}

Si \(X_1,...,X_n\) es \(iid\) con distribución \(F\) cualquiera,
entonces, para \(n\) tendiendo a infinito,

\begin{equation}
X^*_n=M_F= max(X_1,...,X_n)\;tiende\;a\;M_F\label{eq:Xast}
\end{equation}

Observación:

El resultado anterior vale incluso si \(M_F\) es infinito, pero si
\(M_F\) es finito, como \(X^*n - M_F\) tiende a cero, por analogía con
el Teorema Central del Límite para promedios, buscaríamos una sucesión
\(c_n>0\) y que tienda a cero de modo tal que \((X^*n- M_F )/ c_n\)
tienda a una distribución no-degenerada y de allí surge buscar la DEA.

\hypertarget{teorema-5}{%
\subparagraph{Teorema 5}\label{teorema-5}}

Si \(F\) es una distribución con \(M_F\) finito, y para \(X\) con
distribución \(F\) se cumple que

\[
P(X=M_F)>0 
\]

entonces \(F\) NO admite DEA.

Observación:

Si \(F\) es \(Bin(m,p)\), \(M_F=m\). Si \(X\) tiene distribución \(F\),
entonces \(P( X=M_F)= P( X=m)= p_m>0\), asi que la distribucion
\(Bin(m,p)\) NO admite DEA, no se puede aproximar la distribución del
máximo de una muestra \(iid\) de variables \(Bin(m,p)\).

El Teorema anterior es un caso particular del próximo.

\hypertarget{teorema-6}{%
\subparagraph{Teorema 6}\label{teorema-6}}

Si \(F\) es una distribución con \(M_F\) finito o infinito que admite
DEA, y \(X\) tiene distribución \(F\), entonces el límite cuando \(t\)
tiende a \(M_F\) por izquierda de \(P(X>t)/P(X \geq t)\) debe ser 1.

Observación:

\begin{itemize}
\item Si $F$ es una distribución de Poisson de parámetro $\lambda>0$, $M_F$ es infinito. 
\item Si $k$ es un natural, entonces:
\begin{eqnarray}
\frac{P(X>k)}{P(X\geq k)} &=& \frac{P(X \geq k+1)}{P(X\geq k)} \\ \nonumber
&=& 1-\frac{P(X=k)}{P(X \geq k)} \approx 1-\left(1- \frac{\lambda}{k}\right) 
\end{eqnarray}
que tiende a $0$ cuando $k$ tiende a infinito, por lo cual $F$ NO admite DEA, o sea que no se puede aproximar el máximo de una sucesión $iid$ de variables de Poisson.
\end{itemize}

Observación:

El Teorema 6 brinda una condición NECESARIA pero NO SUFICIENTE para DEA.
Un ejemplo de ello lo aportó Von Mises, mostrando que la distribución

\[F(x)= 1- e^{(-x-sen(x))}\] cumple con la condicion del Teorema 6 pero
no admite DEA.

\hypertarget{definiciuxf3n-4-distribuciuxf3n-max-estables}{%
\subsubsection{Definición 4: Distribución
max-estables}\label{definiciuxf3n-4-distribuciuxf3n-max-estables}}

Si dada una \(F\) distribución, \(X\) con distribución \(F\), \(k\)
natural arbitrario y \(X_1,...,X_k\) es \(iid\) con distribución \(F\),
existen reales \(a_k\), \(b_k\) tales que \(max(X_1,...,X_k)\) tiene la
misma distribución que \(a_k X+ b_k\), \(F\) se dice
\textit{max-estable}.

El Teorema FTG resulta de superponer los dos siguientes teoremas:

\hypertarget{teorema-7}{%
\subparagraph{Teorema 7}\label{teorema-7}}

\begin{itemize}
  \item[a)] Si $F$ admite $DEA\;H$, entonces $H$ es max-estable.
  \item[b)] Si $H$ es max-estable, es la DEA de sí misma.
\end{itemize}

\hypertarget{teorema-8}{%
\subparagraph{Teorema 8}\label{teorema-8}}

Una distribución es max-estable si y solo si es
extremal\footnote{O sea Gumbel, Weibull o Fréchet}. El Teorema 7 es
bastante intuitivo y análogo a los teoremas de Lévy sobre distribuciones
estables en aproximaciones asintóticas de las distribuciones de sumas.
Para el Teorema 8 haremos enseguida un ejercicio sencillo que nos
ayudará a hacerlo creíble. Luego precisaremos, para terminar con esta
parte, cómo son las distribuciones que tienen por DEA cada uno de los
tres tipos de distribuciones extremales. Para eso precisamos recordar
algunas definiciones, como la siguiente.

Obsrvación:

Si \(F\) y \(G\) son dos distribuciones, tienen colas equivalentes si
\(M_F=M_G\) y cuando \(t\) tiende a \(M_F\) por izquierda,
\((1-F(t))/(1-G(t))\) tiende a un valor \(c>0\). Recordando ahora cómo
se calcula la distribución del máximo de dos variables independientes,
es muy sencillo calcular la distribución del \(max\{X,Y\}\), cuando
\(X\) e \(Y\) son independientes y cada una de ellas es una distribución
extremal.

Se tiene el siguiente resultado:

\begin{longtable}[]{@{}
  >{\raggedright\arraybackslash}p{(\columnwidth - 4\tabcolsep) * \real{0.2500}}
  >{\raggedright\arraybackslash}p{(\columnwidth - 4\tabcolsep) * \real{0.2500}}
  >{\raggedright\arraybackslash}p{(\columnwidth - 4\tabcolsep) * \real{0.5000}}@{}}
\toprule\noalign{}
\begin{minipage}[b]{\linewidth}\raggedright
\(X\)
\end{minipage} & \begin{minipage}[b]{\linewidth}\raggedright
\(Y\)
\end{minipage} & \begin{minipage}[b]{\linewidth}\raggedright
\(max(X,Y)\)
\end{minipage} \\
\midrule\noalign{}
\endhead
\bottomrule\noalign{}
\endlastfoot
\textcolor{red}{Weibull} & \textcolor{red}{Weibull} &
\textcolor{red}{Weibull} \\
\textcolor[rgb]{0.0,0.5,0.0}{Weibull} &
\textcolor[rgb]{0.0,0.5,0.0}{Gumbel} &
\textcolor[rgb]{0.0,0.5,0.0}{Cola equivalente Gumbel} \\
\textcolor{blue}{Weibull} & \textcolor{blue}{Fréchet} &
\textcolor{blue}{Fréchet} \\
\textcolor[rgb]{0.0,0.5,0.0}{Gumbel} &
\textcolor[rgb]{0.0,0.5,0.0}{Weibull} &
\textcolor[rgb]{0.0,0.5,0.0}{Cola equivalente Gumbel} \\
\textcolor{red}{Gumbel} & \textcolor{red}{Gumbel} &
\textcolor{red}{Gumbel} \\
\textcolor{blue}{Gumbel} & \textcolor{blue}{Fréchet} &
\textcolor{blue}{Cola equivalente Fréchet} \\
\textcolor{blue}{Fréchet} & \textcolor{blue}{Weibull} &
\textcolor{blue}{Fréchet} \\
\textcolor{blue}{Fréchet} & \textcolor{blue}{Gumbel} &
\textcolor{blue}{Cola equivalente Fréchet} \\
\textcolor{red}{Fréchet} & \textcolor{red}{Fréchet} &
\textcolor{red}{Fréchet} \\
\end{longtable}

\textcolor{red}{\rule{1em}{1em} Las extremales son max-estables: tomar máximos de dos del mismo tipo queda en el mismo tipo.}

\textcolor[rgb]{0.0,0.5,0.0}{\rule{1em}{1em} Gumbel es más pesada que Weibull. En la cola, que es lo que cuenta para máximos, prima Gumbel.}

\textcolor{blue}{\rule{1em}{1em} Fréchet es más pesada que Gumbel y mucho más pesada que Weibull.}
\vspace{1cm}

Además, de la tabla se deduce que

\hypertarget{teorema-9}{%
\subparagraph{Teorema 9}\label{teorema-9}}

Si \(X_1,...,X_n\) independientes y cada \(X_i\) tiene uno de los tres
tipos de distribución extremal, entonces la distribución del
\(max(X_1,...,X_n)\) es:

\begin{itemize}
\item[a)] Cola equivalente a Fréchet, si alguna de las variables es Fréchet y alguna otra es Gumbel.
\item[b)]  Fréchet, si alguna es Fréchet y ninguna es Gumbel.
\item[c)]  Cola equivalente Gumbel ninguna es Fréchet pero algunas son Gumbel y otras Weibull.
\item[d)] Gumbel si todas son Gumbel.
\item[e)]  Weibull si todas son Weibull.
\end{itemize}

Observación:

Si \(F\) es una distribución, se dice que tiene
\textit{cola de variación regular de orden} \(-\alpha\), para
\(\alpha \geq 0\), si para todo \(t>0\), \((1-F(tx))/(1-F(x))\) tiende a
\(t^{-\alpha}\) si \(x \rightarrow \infty\). Para abreviar se dirá que
\(F\) es \(R_{-\alpha}\). Por ejemplo, para \(\alpha=3\), un caso de una
tal \(F\) es \(F(u)=1- 1/u^3\).

Por otra parte se dice que \(L\) es una
\textit{función de variación lenta} si, para todo \(t>0\),
\(L(tx)/L(x)\) tiende a 1 cuando \(x \rightarrow \infty\). Por ejemplo,
\(L(u)=log(u)\).

\newpage

\hypertarget{definiciuxf3n-4-dominio-de-atracciuxf3n-maximal}{%
\subsubsection{Definición 4: Dominio de atracción
maximal}\label{definiciuxf3n-4-dominio-de-atracciuxf3n-maximal}}

Si \(H\) es una distribución extremal (Gumbel, Weibull o Fréchet) su
Dominio de Atracción Maximal (\(DAM(H)\)) está constituído por todas las
distribuciones \(F\) que tienen \(DEA\;H\).

\hypertarget{teorema-9-dam-de-la-fruxe9chet}{%
\subparagraph{Teorema 9: DAM de la
Fréchet}\label{teorema-9-dam-de-la-fruxe9chet}}

\(F\) pertenece a la DAM de \(\Phi_{\alpha}\) si y sólo si
\(1-F(x)=x-\alpha L(x)\) para alguna \(L\) de variación lenta, lo cual
es equivalente a decir que \(F\) es \(R_{-\alpha}\).

\hypertarget{corolario-1-dam-de-la-fruxe9chet}{%
\subparagraph{Corolario 1: DAM de la
Fréchet}\label{corolario-1-dam-de-la-fruxe9chet}}

Si \(F\) es una distribución con densidad \(f\) que cumple que
\(xf(x)/(1-F(x))\) tiende a \(\alpha\) cuando \(x \rightarrow \infty\)
se dice que \(F\) cumple la Condición de Von Mises I. En tal caso, \(F\)
pertenece a la DAM de \(\Phi_{\alpha}\) y mas aún, la DAM de
\(\Phi_{\alpha}\) son todas las distribuciones que tienen cola
equivalente a alguna distribución que cumpla la Condición de Von Mises
I. Del DAM Fréchet y Teorema 1, surge lo siguiente.

\hypertarget{teorema-10-dam-de-la-weibull}{%
\subparagraph{Teorema 10: DAM de la
Weibull}\label{teorema-10-dam-de-la-weibull}}

\begin{itemize}
\item [a)] $F$ pertenece a la DAM de $\Psi_{\alpha}$ si y solo si $M_F$ es finito y además $$1-F(M_F -1/x)=x^{-\alpha} L(x)$$ para alguna
$L$ de variación lenta, es decir que pertenece a $R_{-\alpha}$. Observar que con el cambio de variable $u=M_F -1/x$,
resulta que $1-F(u)=(^{-}MF -u)^{\alpha} L(1/(M_F -u))$ para alguna $L$ de variación lenta, para $u< M_F$. Además puede tomarse $d_n= M_F$ y $c_n= n-\alpha$.
\item [b)] Si $F$ distribución con densidad $f$ positiva en $(a,M_F)$ para algun $a< M_F$ y $(M_F -x)f(x)/(1-F(x))$ tiende a $\alpha$ cuando $x\rightarrow M_F$, se dice que $F$ cumple la Condición de Von Mises II. En tal caso $F$ pertenece a la DAM de $\Psi_{\alpha}$ y mas aún, la DAM de $\Psi_{\alpha}$ son todas las distribuciones que tienen cola equivalente a alguna distribución que cumpla la Condición de Von Mises II.
\end{itemize}

\hypertarget{teorema-11-dam-de-la-gumbel}{%
\subparagraph{Teorema 11: DAM de la
Gumbel}\label{teorema-11-dam-de-la-gumbel}}

Una distribución \(F\) se dice una Función de Von Mises con función
auxiliar \(h\) si existe \(a < M_F\) (\(M_F\) puede ser finito o
infinito) tal que para algún \(c>0\) se tiene

\[
1-F(x)= c\;exp^{{- \int_a^X \frac{1}{h(t)} dt}},
\]

con \(h\) positiva, con densidad \(h^\prime\) y \(h^\prime(x)\)
tendiendo a \(0\) para \(x\rightarrow M_F\) Se tiene entonces que la
\(DAM\) de \(\Lambda\) son todas las distribuciones que tienen cola
equivalente a alguna distribución que sea una Función de Von Mises.
Básicamente, se trata de colas más livianas que cualquier expresión del
tipo \(1/x^k\), más aún, con decaimiento \textit{del tipo exponencial},
en el sentido preciso siguiente: si como en el Teorema 11

\(1-F(x)= c\;exp^{{- \int_a^X \frac{1}{h(t)} dt}}\), entonces se tiene
\(1-F(x)= c\;exp^{-(x-a)/h(x)}\), donde la función auxiliar \(h\) es
no-decreciente y con asíntota horizontal.

Además, \(d_n\) y \(c_n\) suelen involucrar expresiones logarítmicas.
Más concretamente, \(dn = F^{-1}(1-1/n)\), \(c_n = h(d_n)\), donde
\(F^{-1}\) es la inversa generalizada (o función cuantil), definida por
\(F^{-1}(p)= inf\{t / F(t)\geq p\}\), para \(0<p<1\).

\hypertarget{corolario-2}{%
\subsubsection{Corolario 2 :}\label{corolario-2}}

Si \(F\) pertenece al \(DAM\) Gumbel, \(M_F\) es infinito, y se
considera \(X\) con distribucion \(F\), entonces \(E(X+k)\) es finito
para todo \(k\) natural. Los resultados antes vistos nos permiten
reconocer que distribuciones tienen \(DEA\) y si la tienen, cual es.
Cierran el tema. Adicionalmente, permiten ver con mucha precision que el
quid de esta teoría es el comportamiento de las colas de las
distribuciones, que Fréchet corresponde a las colas más pesadas, luego
la Gumbel y finalmente Weibull. Para terminar el capítulo presentaremos
la distribución de valores extremos
generalizada\footnote{GEV, por sus siglas en inglés.}, que es una forma
de compactar en una unica fórmula las tres distribuciones extremales,
debida a Jenkinson-Von Mises.

\hypertarget{definiciuxf3n-5-gev}{%
\subsubsection{Definición 5: GEV}\label{definiciuxf3n-5-gev}}

\newpage

\hypertarget{referencias-bibliogruxe1ficas}{%
\section{Referencias
bibliográficas}\label{referencias-bibliogruxe1ficas}}

\vspace{1cm}
\setlength{\parindent}{-0.2in}
\setlength{\leftskip}{0.2in}

\hypertarget{refs}{}
\begin{CSLReferences}{1}{0}
\leavevmode\vadjust pre{\hypertarget{ref-notas_curso}{}}%
Perera, Gonzalo, Angel Segura, y Carolina Crisci. 2021. \emph{Curso de
estadística de datos extremales, cap. 1 a cap. 5}.

\leavevmode\vadjust pre{\hypertarget{ref-evd}{}}%
Stephenson, A. G. 2002. {«evd: Extreme Value Distributions»}. \emph{R
News} 2 (2): 0. \url{https://CRAN.R-project.org/doc/Rnews/}.

\end{CSLReferences}

\end{document}
