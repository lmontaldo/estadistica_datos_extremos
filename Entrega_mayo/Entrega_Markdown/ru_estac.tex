% Options for packages loaded elsewhere
\PassOptionsToPackage{unicode}{hyperref}
\PassOptionsToPackage{hyphens}{url}
\PassOptionsToPackage{dvipsnames,svgnames,x11names}{xcolor}
%
\documentclass[
  12pt]{article}
\usepackage{amsmath,amssymb}
\usepackage{iftex}
\ifPDFTeX
  \usepackage[T1]{fontenc}
  \usepackage[utf8]{inputenc}
  \usepackage{textcomp} % provide euro and other symbols
\else % if luatex or xetex
  \usepackage{unicode-math} % this also loads fontspec
  \defaultfontfeatures{Scale=MatchLowercase}
  \defaultfontfeatures[\rmfamily]{Ligatures=TeX,Scale=1}
\fi
\usepackage{lmodern}
\ifPDFTeX\else
  % xetex/luatex font selection
\fi
% Use upquote if available, for straight quotes in verbatim environments
\IfFileExists{upquote.sty}{\usepackage{upquote}}{}
\IfFileExists{microtype.sty}{% use microtype if available
  \usepackage[]{microtype}
  \UseMicrotypeSet[protrusion]{basicmath} % disable protrusion for tt fonts
}{}
\makeatletter
\@ifundefined{KOMAClassName}{% if non-KOMA class
  \IfFileExists{parskip.sty}{%
    \usepackage{parskip}
  }{% else
    \setlength{\parindent}{0pt}
    \setlength{\parskip}{6pt plus 2pt minus 1pt}}
}{% if KOMA class
  \KOMAoptions{parskip=half}}
\makeatother
\usepackage{xcolor}
\usepackage{graphicx}
\makeatletter
\def\maxwidth{\ifdim\Gin@nat@width>\linewidth\linewidth\else\Gin@nat@width\fi}
\def\maxheight{\ifdim\Gin@nat@height>\textheight\textheight\else\Gin@nat@height\fi}
\makeatother
% Scale images if necessary, so that they will not overflow the page
% margins by default, and it is still possible to overwrite the defaults
% using explicit options in \includegraphics[width, height, ...]{}
\setkeys{Gin}{width=\maxwidth,height=\maxheight,keepaspectratio}
% Set default figure placement to htbp
\makeatletter
\def\fps@figure{htbp}
\makeatother
\setlength{\emergencystretch}{3em} % prevent overfull lines
\providecommand{\tightlist}{%
  \setlength{\itemsep}{0pt}\setlength{\parskip}{0pt}}
\setcounter{secnumdepth}{5}
% definitions for citeproc citations
\NewDocumentCommand\citeproctext{}{}
\NewDocumentCommand\citeproc{mm}{%
  \begingroup\def\citeproctext{#2}\cite{#1}\endgroup}
\makeatletter
 % allow citations to break across lines
 \let\@cite@ofmt\@firstofone
 % avoid brackets around text for \cite:
 \def\@biblabel#1{}
 \def\@cite#1#2{{#1\if@tempswa , #2\fi}}
\makeatother
\newlength{\cslhangindent}
\setlength{\cslhangindent}{1.5em}
\newlength{\csllabelwidth}
\setlength{\csllabelwidth}{3em}
\newenvironment{CSLReferences}[2] % #1 hanging-indent, #2 entry-spacing
 {\begin{list}{}{%
  \setlength{\itemindent}{0pt}
  \setlength{\leftmargin}{0pt}
  \setlength{\parsep}{0pt}
  % turn on hanging indent if param 1 is 1
  \ifodd #1
   \setlength{\leftmargin}{\cslhangindent}
   \setlength{\itemindent}{-1\cslhangindent}
  \fi
  % set entry spacing
  \setlength{\itemsep}{#2\baselineskip}}}
 {\end{list}}
\usepackage{calc}
\newcommand{\CSLBlock}[1]{\hfill\break\parbox[t]{\linewidth}{\strut\ignorespaces#1\strut}}
\newcommand{\CSLLeftMargin}[1]{\parbox[t]{\csllabelwidth}{\strut#1\strut}}
\newcommand{\CSLRightInline}[1]{\parbox[t]{\linewidth - \csllabelwidth}{\strut#1\strut}}
\newcommand{\CSLIndent}[1]{\hspace{\cslhangindent}#1}
\ifLuaTeX
\usepackage[bidi=basic]{babel}
\else
\usepackage[bidi=default]{babel}
\fi
\babelprovide[main,import]{spanish}
% get rid of language-specific shorthands (see #6817):
\let\LanguageShortHands\languageshorthands
\def\languageshorthands#1{}
\usepackage{makeidx}
\makeindex
\usepackage{graphicx}
\usepackage{tikz}
\usepackage{atbegshi}
\usepackage{amsthm}
\newtheorem{definition}{Definición}[section]

\AtBeginDocument{
    \AtBeginShipoutNext{
        \AtBeginShipoutUpperLeft{
            \put(\dimexpr\paperwidth/2-\textwidth/2\relax, -650){
                \makebox[\textwidth]{
                    \includegraphics[width=0.6\textwidth]{cure_udelar.png}  % Adjust width as needed
                    \hfill
                    \includegraphics[width=0.405\textwidth]{logoMEDIA.jpeg} % Make it 90% smaller
                }
            }
        }
    }
}

\usepackage{amsthm}
\usepackage{xcolor}
\ifLuaTeX
  \usepackage{selnolig}  % disable illegal ligatures
\fi
\usepackage{bookmark}
\IfFileExists{xurl.sty}{\usepackage{xurl}}{} % add URL line breaks if available
\urlstyle{same}
\hypersetup{
  pdftitle={Entrega: curso de datos extremales},
  pdfauthor={Laura Montaldo, CI: 3.512.962-7},
  pdflang={es},
  colorlinks=true,
  linkcolor={black},
  filecolor={Maroon},
  citecolor={Blue},
  urlcolor={Blue},
  pdfcreator={LaTeX via pandoc}}

\title{Entrega: curso de datos extremales}
\author{Laura Montaldo, CI: 3.512.962-7}
\date{2024-05-01}

\begin{document}
\maketitle

\newtheorem{theorem}{Teorema}
\newtheorem{mydefinition}{Definición}
\newtheorem{observation}{Observación}
\newtheorem{Corolario}{Corolario}

\newpage

\thispagestyle{empty}

\maketitle

\newpage

\tableofcontents

\newpage

\section{Resumen}\label{resumen}

Your abstract goes here.

\newpage

\section{Motivación y objetivo del
estudio}\label{motivaciuxf3n-y-objetivo-del-estudio}

Siguiendo a Crisci et~al. (\citeproc{ref-notas_curso}{2021}), se dice
que tenemos datos extremos cuando cada dato corresponde al máximo o
mínimo de varios registros. Son un caso particular de evento raro o gran
desviación respecto a la media. Es por este motivo que en una gran
variedad de dominios disciplinares suele ser de gran interés el estudio
de datos extremos. Además, admiten diversos enfoques. La teoría clásica
de estadística de datos extremos se basa en los trabajos de Fréchet,
Gumbel, Weibull, Fisher, Tippett, Gnedenko, entre otros. En este
estudio, el foco va a estar puesto en esquemas que extienden a las
distribuciones extremales clásicas. En particular, se va a emplear el
método Picos sobre el Umbral
(POT)\footnote{Por sus siglas en inglés relaticas a  Peaks over Thresholds.}.

En este marco analítico, se pretende desarrollar un indicador de
posibles crisis bursátiles a partir del índice \(S\&P\;500\). En este
sentido, los índices de \(S\&P\) son una familia de índices de renta
variable\footnote{En inglés se llaman equity indices} diseñados para
medir el rendimiento del mercado de acciones en Estados Unidos que
cotizan en bolsas estadounidenses. Esta familia de índices está
compuesta por una amplia variedad de índices basados en tamaño, sector y
estilo. Los índices están ponderados por el criterio
\textit{float-adjusted market capitalization} (FMC). Además, se disponen
de índices ponderados de manera equitativa y con límite de
capitalización de mercado, como es el caso del \(S\&P\:500\). Este este
sentido, el \(S\&P 500\) entraría en el conjunto de índices ponderados
por capitalización bursátil ajustada a la flotación (ver
\href{https://www.spglobal.com/spdji/en/methodology/article/sp-us-indices-methodology/}{\textcolor{blue}{$S\&P$ Dow Jones Indices}}).
El mismo mide el rendimiento del segmento de gran capitalización del
mercado estadounidense. Es considerado como un indicador representativo
del mercado de renta variable de los Estados Unidos, y está compuesto
por 500 empresas constituyentes.

Se busca crear un indicador de una posible crisis bursátil denominado
Índice de Variación de Precios (IVP). Como variable de referencia de
toma la relación de precios al cierre de ayer sobre la de hoy

\begin{equation}
IVP_t=\left ( \frac{Precio_{t-1}}{Precio_t}\right ) -1,\quad\text{para}\; t=1,...,T 
\end{equation} \vspace{0.5cm}

Interpretación del Indicador:

\begin{itemize}
\item Si el $IVP_t$    $\leq$ 0, el precio de cierre de hoy es mayor o igual que el de ayer, lo cual podría ser considerado una señal positiva.
\item Si el $IVP_t$ > 0, el precio de cierre de hoy es menor que el de ayer, lo cual podría considerarse una señal de alerta.
\end{itemize}

\newpage

En las siguiente figuras se muestra la evolución histórica desde la
fecha 03/01/1928 hasta 08/12/2023 del precio al cierre del día del
indicador S\&P 500.

\phantomsection\label{refs}
\begin{CSLReferences}{1}{0}
\bibitem[\citeproctext]{ref-notas_curso}
Crisci, C., Perera, G., \& Segura, A. (2021). \emph{Curso de estadística
de datos extremales, cap. 1 a cap. 5}.

\bibitem[\citeproctext]{ref-enders}
Enders, W. (2014). \emph{Applied Econometric Time Series}. Wiley.

\bibitem[\citeproctext]{ref-kpss}
Shin, Y., Kwiatkowski, D., Schmidt, P., \& Phillips, P. C. B. (1992).
Testing the Null Hypothesis of Stationarity Against the Alternative of a
Unit Root: How Sure Are We That Economic Time Series Are Nonstationary?
\emph{Journal of Econometrics}, \emph{54}(1-3), 159-178.

\bibitem[\citeproctext]{ref-evd}
Stephenson, A. G. (2002). evd: Extreme Value Distributions. \emph{R
News}, \emph{2}(2), 0. \url{https://CRAN.R-project.org/doc/Rnews/}

\end{CSLReferences}

\end{document}
